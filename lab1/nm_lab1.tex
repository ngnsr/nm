\documentclass[a4paper, 12pt]{article}
\usepackage{graphicx}
\usepackage[utf8]{inputenc}
\usepackage[ukrainian]{babel}
\usepackage{amsmath}
\usepackage{fancyhdr}
\usepackage{geometry}
\geometry{top=2cm, bottom=2cm, left=3cm, right=1.5cm}
\usepackage[colorlinks=true,linkcolor=blue]{hyperref}%
\begin{document}

\begin{titlepage}
	\begin{center}
		\Large
		\textbf{Київський національний університет імені Тараса Шевченка} \\
		Факультет комп'ютерних наук та кібернетики \\

		\vspace{6cm}

		\textbf{\LARGE ЗВІТ ДО ЛАБОРАТОРНОЇ РОБОТИ №1} \\[0.5cm]
		\textbf{З дисципліни ``Чисельні методи''} \\[0.5cm]
		\textbf{Тема: Розв'язування нелінійних рівнянь} \\

		\vfill

		\hspace{7cm} Виконав студент 3-го курсу \\
		\hspace{7cm} групи ТТП-31 \\
		\hspace{7cm} Рісенгін Владислав \\
		\vspace{2cm}

		Київ-2024
	\end{center}
\end{titlepage}

\newpage

\hbadness=99999

\section{Постановка задачі}

Варіант № 7.

Знайти розв'язок рівняння $x^4 + x^3 - 6x^2 + 20x - 16 = 0$ методами простої ітерації та релаксації.

\begin{figure}[h]
	\centering
	\includegraphics[width=1\linewidth]{plot.png}
	\caption{Графік рівняння}
\end{figure}
\newpage

\section{Вступ}

Метою цієї лабораторної роботи є вивчення методів розв'язування нелінійних рівнянь, які дають можливість вирішення багатьох наукових та інженерних задач. У процесі виконання роботи будуть досліджені та реалізовані метод простої ітерації та метод релаксації для знаходження коренів нелінійних рівнянь.


% \newpage

\section{Методи які застосовувались у ході розвʼязання}
Мова програмування: Python \\[0.25ex]

$f(x) = x^4 + x^3 + -6x^2 + 20x - 16 = 0$ \\[0.25ex]

$\phi(x) = x + (x^4 + x^3 + -6x^2 + 20x - 16) * \psi(x)= 0$ \\[0.25ex]

$\psi(x) =  \frac{1}{x - 24}$
\\[0.35ex]

$f'(x) = 4x^3 + 3x^2 + -12x + 20$ \\[0.25ex]

\section{Задача}
Знайти найменший додатній корінь нелінійного рівняння  \\ [0.25ex]
$x^4 + x^3 + -6x^2 + 20x - 16 = 0$ \\
методом простої ітерації та методом релаксації з точністю $\epsilon = 10^{-4}$.

Знайти апріорну та апостеріорну оцінку кількості кроків. \\
Початковий проміжок та початкове наближення обрати однакове для обох методів (якщо) це можливо. Порівняти результати методів між собою. \\

\subsection{Метод простої ітерації}

На проміжку $[0; 1.4]$ $f(x)$ монотонно зростає, а також має різні знаки на кінцях. $f'(x)$ на цьому ж проміжку є додатньою
\\[0.25ex]

Оберемо проміжок $[a;b] = [0; 1.4]$ \\[0.25ex]

Знайдемо $x_{0} = (a_{0} + b_{0})/2$ , де $a_{0} = a, b_{0} = b$ \\[0.25ex]

$x_{0} = (0 + 1.4)/2 = 0.7$ \\[0.25ex]

$ \sigma = 0.7$ \\[0.25ex]

Підберемо таку функцію $\psi(x)$ щоб модуль похідної від функції $\phi(x) = x + f(x)* \psi(x)$ був < 1 на проміжку $[a;b]$ \\[0.25ex]
$\psi(x) =  \frac{1}{x - 24}$ \\
Обчислимо $\phi(x)$ за формулою: \\
$\phi(x) = x + (x^4 + x^3 + -6x^2 + 20x - 16) * \psi(x)= 0$ \\[0.25ex]


Для визначення q знайдемо критичні точки $\phi'(x) = 0$  на $[0 ; 1.4]$ -- 0.73011,  \\
$\phi''(0.73011) \approx -1.011$, отже це точка локального максимума.  \\
$\phi(0.73011) \approx 0.39 $ \\


\begin{figure}[h]
	\centering
	\includegraphics[width=0.8\linewidth]{der_phi.png}
	\caption{$\phi'(x)$}
	\label{fig:der_phi}
\end{figure}

Також перевіримо графічно див рис. \ref{fig:der_phi}

$q = 0.4 $ \\

Перевіримо формулу (12) \\

$|\phi(x_{0}) -x_{0}| <= (1-q)\sigma$ \\[0.25ex]

$\phi(x_{0}) = 0.886991, x_{0} = 0.7$ , $(1-0.4)*0.7 = 0.225$ => $0.186991 <= 0.42$ \\[0.25ex]

Знайдемо апріорну оцінку: \\
\[

	n \leq \left[ \frac{\ln\left( \frac{\phi(x_0 - x_0)}{(1-q) * \epsilon} \right)}{\ln(1/q)} \right] + 1 =
	\left[ \frac{\frac{\ln(0.886991 - 0.7)}{(1-0.4) * 10^{-4}}}{\ln(1/0.4)} \right] + 1 = 9
\]

Апостеріорна оцінка обчислюється за наступною формулою:
\[
	\left| x_n - x_* \right| \leq \frac{q}{1 - q} \left| x_n - x_{n-1} \right|.
\]
\newpage

\subsection{Метод релаксації}
На проміжку $[0; 1.4]$ $f(x)$ монотонно зростає, а також має різні знаки на кінцях. $f'(x)$ на цьому ж проміжку є додатньою
\\[0.25ex]


Обчисливши першу похідну і прирівнявши до нуля, отримаємо дві критичні точки $0.78$ і $-1.28$. \\

Якщо перевірити за другою похідною - це будуть локальний мінімум, і локальний максимум відповідно.
Отже мінімум обчислим за наступною формулою :

$m_{1} = min|f'(x)| = |f'(0.78)| \approx 14.364$
\\[0.25ex]

А максимум за f'(a) і f'(b) адже функція зростає на (0.78; 1.4) і спадає на [0; 0.78) (через знак похідної):

$M_{1} = max|f'(x)| = |f'(0)| = 20$
\\[0.25ex]

\begin{figure}[h]
	\centering
	\includegraphics[width=0.8\linewidth]{der_f.png}
	\caption{Похідна функції f}
	\label{fig:der_f}
\end{figure}

Також перевіримо графічно: див рис. \ref{fig:der_f}\\


$\tau_{opt} = \frac{2}{M_{1} + m_{1}} = \frac{2}{34} = 0.0059$
\\[0.25ex]


Умова збіжності $-2 < \tau f'(x) < 0$ виконується.

Додаткові обчислення:

$q = \frac{M_{1} - m_{1}}{M_{1} + m_{1}} = \frac{20 - 14}{20 + 14} = 0.176$
\\[0.25ex]

$z_{0} = \frac{0 + 1.4}{2} = 0.7$
\\[0.25ex]

Знайдемо апріорну оцінку: \\
\[
	n \leq [\frac{ln(|z_{0}|/\epsilon}{ln(1/q)}] + 1 \leq [\frac{8.8536}{1.734}] + 1 = 6
\]

Апостеріорна оцінка обчислюється за наступною формулою: \\
\[
	\left| z_n \right| \leq  q^n * \left| z_0 \right|
\]

\newpage

\section{Результати роботи програми}

\begin{figure}[h]
	\centering
	\includegraphics[width=0.8\linewidth]{iteration.png}
	\caption{таблиця результатів для методу простої ітерації}
\end{figure}

\begin{figure}[h]
	\centering
	\includegraphics[width=0.8\linewidth]{relaxation.png}
	\caption{Таблиця результатів для методу релаксації}
\end{figure}

\begin{figure}[h]
	\centering
	\includegraphics[width=0.7\linewidth]{result.png}
	\caption{Таблиця порявняння}
\end{figure}


\newpage
\section{Висновок}
Ми отримали наближений корінь нелінійного рівняння двома різними способами:␍ \\
\begin{align*}
	\text{Методом простої ітерації:} & \quad x^* = 0.9999740300057031 \\
	\text{Методом релаксації:}       & \quad x^* = 0.9999989452870753
\end{align*}
Проте могли зупинитись після 8 кроку для методу простої ітерації так як точність вже досягається, \\
для методу релаксації точність досягається після 6 кроку - як і за апріорною оцінкою. \\

Отже, у даному випадку метод релаксації досягнув набагато кращої точності навіть при меншій кількості кроків. \\
З іншої ж сторони різниця не така вже й велика, адже алгоритм на 9 та 6 кроків - це те що може бути швидко обраховане навіть вручну. \\
Обидва методи відпрацювали чудово.

\newpage
\section{Вихідний код програми}
Посилання на \href{https://github.com/ngnsr/nt_lab1}{GitHub}
\end{document}
